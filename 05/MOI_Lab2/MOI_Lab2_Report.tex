%!TEX TS-program = xelatex
%!TEX encoding = UTF-8 Unicode

\documentclass[14pt,a4paper]{extarticle}

\usepackage{mathtools} % Mathematical tools to use with amsmath
\usepackage{amsfonts} % TeX fonts from the American Mathematical Society
\usepackage{mathspec} % Specify arbitrary fonts for mathematics in XeTeX
\usepackage{fontspec} % Advanced font selection in XeLaTeX and LuaLaTeX
\usepackage{indentfirst} % Indent first paragraph after section header
\usepackage{polyglossia} % Multilingual support for XeLaTeX
\usepackage[
  left=2cm,right=2cm,top=2cm,bottom=2cm,headheight=5mm,headsep=5mm,includehead
]{geometry} % Flexible and complete interface to document dimensions
\usepackage{makeidx} % Standard LaTeX package for creating indexes
\usepackage[
  colorlinks=true, allcolors=blue,
]{hyperref} % Extensive support for hypertext in LaTeX
\usepackage{titlesec} % Select alternative section titles
\usepackage{array} % Extending the array and tabular environments
\usepackage{amsthm} % Typesetting theorems (AMS style)
\usepackage{mathrsfs} % Support for using RSFS fonts in maths
\usepackage{listings} % Typeset source code listings using LaTeX

% latex
\pagestyle{myheadings}

% mathspec
\setmathsfont(Digits,Latin,Greek){Times New Roman}

% fontspec
\defaultfontfeatures{Mapping=tex-text}
\setmainfont{Times New Roman}
\setmonofont[Mapping=tex-text]{Fantasque Sans Mono}
\newfontfamily\cyrillicfont{Times New Roman}
\newfontfamily\cyrillicfontsf[Script=Cyrillic]{Fantasque Sans Mono}
\newfontfamily\cyrillicfonttt[Script=Cyrillic]{Fantasque Sans Mono}

% polyglossia
\setdefaultlanguage{ukrainian}

% titlesec
\titleformat{\section}
  {\normalfont\Large\bfseries\centering}{\thesection. }{0pt}{}
\titleformat{\subsection}
  {\normalfont\large\bfseries\centering}{\thesubsection. }{0pt}{}
\titleformat{\subsubsection}
  {\normalfont\large\bfseries\centering}{\thesubsubsection. }{0pt}{}

% array
\def\arraystretch{1.5}

% amsthm
\newtheorem{theorem}{Теорема}[section]
\theoremstyle{definition}
\newtheorem{definition}{Визначення}[section]

\makeatletter
\def\@maketitle{%
  \newpage
  \begin{center}%
    {\scshape
    Міністерство освіти і науки України \\
    Дніпровський національний університет імені Олеся Гончара \\
    Факультет прикладної математики \\
    Кафедра комп'ютерних технологій \par}
    \let \footnote \thanks
    {\LARGE \@title \par}%
  \end{center}%
  \begin{flushright}
    {\scshape
      \begin{tabular}[t]{p{0.2\textwidth}%
        >{\raggedright\arraybackslash}p{0.3\textwidth}}%
        Виконавець: & \@author \\
        & \@date
      \end{tabular}\par}%
  \end{flushright}}
\makeatother

\title{%
  \uppercase{\textbf{Лабораторна робота №2}}\\
  \large з курсу ``Математичні основи інформатики''\\
  Тема: ``Автоматизація доведення програм, що містять розгалуження за
    допомогою ПЗ Simplify''\\
  Варіант 25}
\author{Панасенко Єгор Сергійович \newline студент групи ПА-17-2}
\date{}

\begin{document}
\sloppy % Stretches spaces to correct align of text
\allowdisplaybreaks % Breaks equations between pages

\maketitle
\thispagestyle{empty}

\section{Постановка задачі}

Написати специфікацію. Написати код програми.

Довести програму двома способами:
\begin{enumerate}
  \item за допомогою теореми про розгалуження;
  \item без застосування теореми про розгалуження.
\end{enumerate}

у двох режимах:
\begin{enumerate}
  \item Вручну у вигляді тексту.
  \item В автоматизованому режимі за допомогою ПЗ simplify.
\end{enumerate}

Індивідуальні завдання:
\begin{enumerate}
  \item умови завдань отримати за номером у папці .\textbackslash{}umovy
  \item для кожної задачі навести доведення "вручну" (на папері) та за
допомогою Simplify.
\end{enumerate}

\textbf{Примітка.} У варіантах завдань із номерами 04 та більше
результат роботи програми - це
\begin{enumerate}
  \item координати шахового поля - у випадку стверджувальної відповіді на
  запитання задачі;

  \item координати шахового поля (0,0) - у випадку спростовної відповіді.
\end{enumerate}

\textbf{Індивідуальний варіант.}
Поле шахової доска визначається парою натуральних чисел,
кожне з яких не перевищує восьми:
перше число - номер вертикалі (при рахунку зліва направо),
друге - номер горизонталі (при рахунку від низу до верху).
Дано натуральні числа $ a, b, c, d, e, f $, кожне з яких не перевищує восьми.

На поле $ (a, b) $ розташована біла фігура,
на поле $ (c, d) $ - чорна. Визначити,
чи може біла фігура піти на поле $ (e, f) $,
не потрапивши при цьому під удар чорної фігури.

Розглянути наступні варіанти поєднань білої і чорної фігур: п) слон і кінь.

\section{Хід виконання}

\subsection{Специфікація}

Напишемо специфікацію.

Cлон рухається тільки по діагоналі.
Нехай маємо безкінечна дошка, у якому $(x,y)$ - це будь яке поле на дошці
описане цілими координатами $x,\ y \in \mathbb{Z}$.
Нехай $(0,0)$ - це центр дошки.
Тоді рух слона з центру дошки можна описати за формулою $|x|=|y|$.
Таким чином, щоб описати рух слона з поля $(a,b)$, потрібно взяти $(a,b)$ за
центр дошки, тоді усі координати будуть $(x-a,y-b)$.
Таким чином щоб перевірити, чи може рухатися слон з поля $(a,b)$ у поле
$(e,f)$, потрібно перевірити, чи виконується рівність $|e-a|=|f-b|$.
Перепишемо це мовою предикатів $abs(e-a)=abs(f-b)$ та розкриємо модулі
$e-a=f-b \lor e-a=b-f$

Кінь завжди рухается на два поля вперед і на одне поле в сторону,
якщо кінь рухається з центру дошки, то це можна описати так:
\[abs(x)=1 \land abs(y)=2 \lor abs(x)=2 \land abs(y)=1\]
Розкриємо модулі:
\[(x=1 \lor x=-1) \land (y=2 \lor y=-2) \lor (x=2 \lor x=-2)
\land (y=1 \lor y=-1)\]
Тепер перевіримо, чи може кінь перейти з $(c,d)$ до $(e,f)$, взявши за
центр поле $(c,d)$:
\begin{multline*}
  (e-c=1 \lor e-c=-1) \land (f-d=2 \lor f-d=-2) \lor \\
  \lor (e-c=2 \lor e-c=-2) \land (f-d=1 \lor f-d=-1)
\end{multline*}

По задачі нам потрібно, щоб слон попав з $(a,b)$ до $(e,f)$,
але не був під ударом коня, тобто кінь не міг попасти з $(c,f)$ до $(e,f)$,
також слон не може попасти на $(e,f)$, коли кінь вже стоїть на $(e,f)$

\begin{align*}
  \begin{split}
    \{Q:\ & 1 \leq a \land a \leq 8 \land 1 \leq b \land b \leq 8 \land 1 \leq c
    \land c \leq 8 \\& \land 1 \leq d \land d \leq 8
    \land 1 \leq e \land e \leq 8 \land 1 \leq f \land f \leq 8\}
  \end{split}\\
  \begin{split}
    S:\ & if\\
    &\quad(e-a=f-b \lor e-a=b-f) \land \lnot (e = c \land f = d) \land \\
    &\quad\quad\land\lnot ((e-c=1 \lor e-c=-1) \land (f-d=2 \lor f-d=-2) \lor \\
    &\quad\quad\quad\lor (e-c=2 \lor e-c=-2) \land (f-d=1 \lor f-d=-1)) \\
    &\quad\quad\to x,y := c, d \\
    &\quad|\lnot((e-a=f-b \lor e-a=b-f) \land \lnot (e = c \land f = d) \land \\
    &\quad|\quad\land\lnot ((e-c=1 \lor e-c=-1) \land (f-d=2 \lor f-d=-2) \lor \\
    &\quad|\quad\quad\lor (e-c=2 \lor e-c=-2) \land (f-d=1 \lor f-d=-1))) \\
    &\quad|\quad\to x,y := 0,0 \\
    &fi
  \end{split}\\
  \begin{split}
    \{R:\ & ((e-a=f-b \lor e-a=b-f) \land \lnot (e = c \land f = d) \land \\
    &\quad\land\lnot ((e-c=1 \lor e-c=-1) \land (f-d=2 \lor f-d=-2) \lor \\
    &\quad\quad\lor (e-c=2 \lor e-c=-2) \land (f-d=1 \lor f-d=-1)) \land \\
    &\quad\land x = c \land y = d) \lor \\
    &\lor(\lnot((e-a=f-b \lor e-a=b-f) \land \lnot (e = c \land f = d) \land \\
    &\quad\land\lnot ((e-c=1 \lor e-c=-1) \land (f-d=2 \lor f-d=-2) \lor \\
    &\quad\quad\lor (e-c=2 \lor e-c=-2) \land (f-d=1 \lor f-d=-1))) \land \\
    &\quad\land x = 0 \land y = 0)\}
  \end{split}
\end{align*}

Нехай
\begin{align*}
  \begin{split}
    CH =& (e-a=f-b \lor e-a=b-f) \land \lnot (e = c \land f = d) \land \\
    &\quad\land\lnot ((e-c=1 \lor e-c=-1) \land (f-d=2 \lor f-d=-2) \lor \\
    &\quad\quad\lor (e-c=2 \lor e-c=-2) \land (f-d=1 \lor f-d=-1))
  \end{split}
\end{align*}

Тоді специфікацію можна записати так:
\begin{align*}
  \begin{split}
    \{Q:\ & 1 \leq a \land a \leq 8 \land 1 \leq b \land b \leq 8 \land 1 \leq c
    \land c \leq 8 \\& \land 1 \leq d \land d \leq 8
    \land 1 \leq e \land e \leq 8 \land 1 \leq f \land f \leq 8\}
  \end{split}\\
  \begin{split}
    S:\ & if\\
    &\quad CH \to x,y := c,d \\
    &\quad|\lnot(CH) \to x,y := 0,0 \\
    &fi
  \end{split}\\
  \begin{split}
    \{R:\ & (CH \land x = c \land y = d)
    \lor (\lnot(CH) \land x = 0 \land y = 0)\}
  \end{split}
\end{align*}


\subsection{Доведення}

\subsection{Доведення з теоремою про розгалуження}
Наведемо теорему про розгалуження.
\begin{theorem}[про розгалуження]
  Розглянемо команду $IF$.
  Нехай предикат $Q$ задовільняє умовам:
  \begin{enumerate}
    \item $Q \Rightarrow BB$
    \item $Q \land B_i \Rightarrow \mathit{wp}(S_i, R)$ для всіх $i$,
    $1 \leq i \leq n$.
  \end{enumerate}
  Тоді і тільки тоді $Q \Rightarrow \mathit{wp}(IF, R)$.
\end{theorem}

\begin{enumerate}
  \item $Q \Rightarrow BB$
  \begin{align*}
    \begin{split}
      &Q \Rightarrow B1 \lor B2 \text{, (за означенням IF)} \\
    \end{split}\\
    \begin{split}
      &Q \Rightarrow CH \lor \lnot CH \\
    \end{split}\\
    \begin{split}
      &Q \Rightarrow TRUE
      \text{, (закон виключення третього: }E1 \lor \lnot E2 = T\text{ )} \\
    \end{split}\\
    \begin{split}
      &\lnot Q \lor TRUE
      \text{, (закон імплікації)} \\
    \end{split}\\
    &TRUE\text{, (закон спрощення $\lor$)}
  \end{align*}

  Таким чином ми довели, що
  \begin{equation}
    Q \Rightarrow BB = TRUE
  \end{equation}

  \item $(\forall i: 1 \leq i \leq 2: Q \land Bi \Rightarrow
  \mathit{wp}(S_i, R))$
  \begin{align*}
    &Q \land CH \Rightarrow  \mathit{wp}(x,y := c,d,R)
      \land Q \land \lnot CH \Rightarrow \mathit{wp}(x,y := 0,0,R))\\\\
    &(Q \land CH \Rightarrow R_{c,d}^{x,y})
      \land (Q \land \lnot CH \Rightarrow R_{0,0}^{x,y})
      \text{, (за озн. присвоювання)} \\\\
    \begin{split}
      &(Q \land CH \Rightarrow (CH \land c = c \land d = d)
        \lor (\lnot(CH) \land c = 0 \land d = 0)) \land \\
      &\land (Q \land \lnot CH \Rightarrow (CH \land 0 = c \land 0 = d)
        \lor (\lnot(CH) \land 0 = 0 \land 0 = 0)) \\
    \end{split}\\\\
    \begin{split}
      &(Q \land CH \Rightarrow (CH \land TRUE \land TRUE)
        \lor (\lnot(CH) \land c = 0 \land d = 0)) \land \\
      &\land (Q \land \lnot CH \Rightarrow (CH \land 0 = c \land 0 = d)
        \lor (\lnot(CH) \land TRUE \land TRUE)) \\
    \end{split}\\\\
    \begin{split}
      &(Q \land CH \Rightarrow CH
        \lor (\lnot(CH) \land c = 0 \land d = 0)) \land \\
      &\land (Q \land \lnot CH \Rightarrow (CH \land 0 = c \land 0 = d)
        \lor \lnot CH) \\
    \end{split}\\\\
    \begin{split}
      &(\lnot(Q \land CH) \lor CH
        \lor (\lnot(CH) \land c = 0 \land d = 0)) \land \\
      &\land (\lnot(Q \land \lnot CH) \lor (CH \land 0 = c \land 0 = d)
        \lor \lnot CH) \text{, (закон імплікації)} \\
    \end{split}\\\\
    \begin{split}
      &(\lnot Q \lor \lnot CH \lor CH
        \lor (\lnot(CH) \land c = 0 \land d = 0)) \land \\
      &\land (\lnot Q \lor CH \lor (CH \land 0 = c \land 0 = d)
        \lor \lnot CH) \text{, (закон де Моргана)} \\
    \end{split}\\\\
    \begin{split}
      &(\lnot Q \lor TRUE
        \lor (\lnot(CH) \land c = 0 \land d = 0)) \land \\
      &\land (\lnot Q \lor TRUE \lor (CH \land 0 = c \land 0 = d))
        \text{, (закон виключення третього)} \\
    \end{split}\\\\
    &TRUE \land TRUE\text{, (закон спрощення $\lor$)} \\\\
    &TRUE
  \end{align*}

  Таким чином ми довели, що
  \begin{equation}
    (\forall i: 1 \leq i \leq 2: Q \land Bi \Rightarrow
      \mathit{wp}(S_i, R)) = TRUE
  \end{equation}
\end{enumerate}

Ми довели, що для нашої задачі $Q \Rightarrow \mathit{wp}(S,R)$

\subsection{Доведення без теореми про розгалуження}

\begin{definition}
  \[\mathit{wp}(IF, R) = (\exists i:1 \leq i \leq n:B_i)\land
    (\forall i: 1 \leq i \leq n: B_i \Rightarrow \mathit{wp}(S_i, R))\]
\end{definition}

Для того, щоб довести програму, потрібно довести, що
$Q \Rightarrow \mathit{wp}(S,R)$ тавтологія.

Спочатку приведемо вираження до виду, який буде легко перевести на мову
Simplify.
\begin{align*}
  &Q \Rightarrow \mathit{wp}(IF,R) \\\\
  &Q \Rightarrow (\exists i:1 \leq i \leq 2:B_i)\land
    (\forall i: 1 \leq i \leq 2: B_i \Rightarrow \mathit{wp}(S_i, R)) \\\\
  &Q \Rightarrow (B1 \lor B2)\land
    (\forall i: 1 \leq i \leq 2: B_i \Rightarrow \mathit{wp}(S_i, R)) \\\\
  &Q \Rightarrow (B1 \lor B2)\land (B_1 \Rightarrow \mathit{wp}(S_1, R))
    \land (B_2 \Rightarrow \mathit{wp}(S_2, R)) \\\\
  \begin{split}
    &Q \Rightarrow (CH \lor \lnot CH)
      \land (CH \Rightarrow \mathit{wp}(x,y := c,d, R))\\
      &\land (\lnot CH \Rightarrow \mathit{wp}(x,y := 0,0, R))\\
  \end{split}\\\\
  &Q \Rightarrow (CH \lor \lnot CH)
    \land (CH \Rightarrow R_{c,d}^{x,y})
    \land (\lnot CH \Rightarrow R_{0,0}^{x,y})\\\\
  \begin{split}
    &Q \Rightarrow (CH \lor \lnot CH)
      \land (CH \Rightarrow (CH \land c = c \land d = d)
        \lor (\lnot(CH) \land c = 0 \land d = 0))\\
      &\land (\lnot CH \Rightarrow (CH \land 0 = c \land 0 = d)
        \lor (\lnot(CH) \land 0 = 0 \land 0 = 0))\\
  \end{split}
\end{align*}

Ми привели до простого виду, якщо підставити $CH$,
то ми отримаємо предикат у якому тільки операції
$\lnot, \land, \lor, =, \Rightarrow$.
Тепер можна доводити вручну.

\begin{align*}
  \begin{split}
    &Q \Rightarrow (CH \lor \lnot CH)
      \land (CH \Rightarrow (CH \land TRUE \land TRUE)
        \lor (\lnot(CH) \land c = 0 \land d = 0))\\
      &\land (\lnot CH \Rightarrow (CH \land 0 = c \land 0 = d)
        \lor (\lnot(CH) \land TRUE \land TRUE))\\
  \end{split}\\\\
  \begin{split}
    &Q \Rightarrow (CH \lor \lnot CH)
      \land (CH \Rightarrow CH
        \lor (\lnot(CH) \land c = 0 \land d = 0))\\
      &\land (\lnot CH \Rightarrow (CH \land 0 = c \land 0 = d)
        \lor \lnot CH)\text{, (закон спрощення $\land$)}\\
  \end{split}\\\\
  \begin{split}
    &\lnot Q \lor (CH \lor \lnot CH)
      \land (\lnot CH \lor CH
        \lor (\lnot(CH) \land c = 0 \land d = 0))\\
      &\land (CH \lor (CH \land 0 = c \land 0 = d)
        \lor \lnot CH)\text{, (закон імплікації)}\\
  \end{split}\\\\
  \begin{split}
    &\lnot Q \lor (TRUE)
      \land (TRUE
        \lor (\lnot(CH) \land c = 0 \land d = 0))\\
      &\land (TRUE \lor (CH \land 0 = c \land 0 = d))
        \text{, (закон виключення третього)}\\
  \end{split}\\\\
  &\lnot Q \lor TRUE \land TRUE \land TRUE
    \text{, (закони спрощення $\lor$)}\\\\
  &\lnot Q \lor TRUE \text{, (закони спрощення $\land$)}\\\\
  &TRUE \text{, (закони спрощення $\lor$)}\\\\
\end{align*}

\subsection{Доведення за допомогою Simplify}
Раніше ми привели предикат до зручної форми для переведення в Simplify:
\begin{align*}
  \begin{split}
    &Q \Rightarrow (CH \lor \lnot CH)
      \land (CH \Rightarrow (CH \land c = c \land d = d)
        \lor (\lnot(CH) \land c = 0 \land d = 0))\\
      &\land (\lnot CH \Rightarrow (CH \land 0 = c \land 0 = d)
        \lor (\lnot(CH) \land 0 = 0 \land 0 = 0))\\
  \end{split}
\end{align*}

Напишемо тепер мовою Simplify:

\begin{lstlisting}[breaklines=true,basicstyle= \ttfamily]
(IMPLIES Q (AND (OR CH (NOT CH))
  (IMPLIES CH (OR (AND CH (EQ c c) (EQ d d))
    (AND (NOT CH) (EQ c 0) (EQ d 0))))
  (IMPLIES (NOT CH) (OR (AND CH (EQ 0 c) (EQ 0 d))
    (AND (NOT CH) (EQ 0 0) (EQ 0 0))))))
\end{lstlisting}

Згадаймо чому дорівнює $CH$:
\begin{align*}
  \begin{split}
    CH =& (e-a=f-b \lor e-a=b-f) \land \lnot (e = c \land f = d) \land \\
    &\quad\land\lnot ((e-c=1 \lor e-c=-1) \land (f-d=2 \lor f-d=-2) \lor \\
    &\quad\quad\lor (e-c=2 \lor e-c=-2) \land (f-d=1 \lor f-d=-1))
  \end{split}
\end{align*}

Напишемо тепер $CH$ мовою Simplify:
\begin{lstlisting}[breaklines=true,basicstyle= \ttfamily]
(AND (OR (EQ (- e a) (- f b)) (EQ (- e a) (- b f)))
  (NOT (AND (EQ e c) (EQ f d)))
  (NOT (OR (AND (OR (EQ (- e c) 1) (EQ (- e c) -1))
                (OR (EQ (- f d) 2) (EQ (- f d) -2)))
           (AND (OR (EQ (- e c) 2) (EQ (- e c) -2))
                (OR (EQ (- f d) 1) (EQ (- f d) -1))))))
\end{lstlisting}

Підставимо це у перший предикат:
\begin{lstlisting}[breaklines=true,basicstyle= \ttfamily\footnotesize]
(IMPLIES Q (AND (OR (AND (OR (EQ (- e a) (- f b)) (EQ (- e a) (- b f)))
(NOT (AND (EQ e c) (EQ f d))) (NOT (OR (AND (OR (EQ (- e c) 1) (EQ (- e c) -1))
(OR (EQ (- f d) 2) (EQ (- f d) -2))) (AND (OR (EQ (- e c) 2) (EQ (- e c) -2))
(OR (EQ (- f d) 1) (EQ (- f d) -1)))))) (NOT (AND (OR (EQ (- e a) (- f b))
(EQ (- e a) (- b f))) (NOT (AND (EQ e c) (EQ f d))) (NOT (OR (AND
(OR (EQ (- e c) 1) (EQ (- e c) -1)) (OR (EQ (- f d) 2) (EQ (- f d) -2)))
(AND (OR (EQ (- e c) 2) (EQ (- e c) -2)) (OR (EQ (- f d) 1) (EQ (- f d) -1))))))))
(IMPLIES (AND (OR (EQ (- e a) (- f b)) (EQ (- e a) (- b f)))
(NOT (AND (EQ e c) (EQ f d))) (NOT (OR (AND (OR (EQ (- e c) 1) (EQ (- e c) -1))
(OR (EQ (- f d) 2) (EQ (- f d) -2))) (AND (OR (EQ (- e c) 2) (EQ (- e c) -2))
(OR (EQ (- f d) 1) (EQ (- f d) -1))))))
(OR (AND (AND (OR (EQ (- e a) (- f b)) (EQ (- e a) (- b f)))
(NOT (AND (EQ e c) (EQ f d))) (NOT (OR (AND (OR (EQ (- e c) 1) (EQ (- e c) -1))
(OR (EQ (- f d) 2)
(EQ (- f d) -2))) (AND (OR (EQ (- e c) 2) (EQ (- e c) -2))
(OR (EQ (- f d) 1) (EQ (- f d) -1))))))
(EQ c c) (EQ d d)) (AND (NOT (AND (OR (EQ (- e a) (- f b))
(EQ (- e a) (- b f))) (NOT (AND (EQ e c) (EQ f d)))
(NOT (OR (AND (OR (EQ (- e c) 1) (EQ (- e c) -1)) (OR (EQ (- f d) 2)
(EQ (- f d) -2))) (AND (OR (EQ (- e c) 2) (EQ (- e c) -2))
(OR (EQ (- f d) 1) (EQ (- f d) -1))))))) (EQ c 0) (EQ d 0))))
(IMPLIES (NOT (AND (OR (EQ (- e a) (- f b)) (EQ (- e a) (- b f)))
(NOT (AND (EQ e c) (EQ f d))) (NOT (OR (AND (OR (EQ (- e c) 1) (EQ (- e c) -1))
(OR (EQ (- f d) 2) (EQ (- f d) -2))) (AND (OR (EQ (- e c) 2)
(EQ (- e c) -2)) (OR (EQ (- f d) 1) (EQ (- f d) -1)))))))
(OR (AND (AND (OR (EQ (- e a) (- f b)) (EQ (- e a) (- b f)))
(NOT (AND (EQ e c) (EQ f d))) (NOT (OR (AND (OR (EQ (- e c) 1)
(EQ (- e c) -1)) (OR (EQ (- f d) 2) (EQ (- f d) -2)))
(AND (OR (EQ (- e c) 2) (EQ (- e c) -2))
(OR (EQ (- f d) 1) (EQ (- f d) -1)))))) (EQ 0 c) (EQ 0 d))
(AND (NOT (AND (OR (EQ (- e a) (- f b)) (EQ (- e a) (- b f)))
(NOT (AND (EQ e c) (EQ f d))) (NOT (OR (AND (OR (EQ (- e c) 1)
(EQ (- e c) -1)) (OR (EQ (- f d) 2) (EQ (- f d) -2)))
(AND (OR (EQ (- e c) 2) (EQ (- e c) -2)) (OR (EQ (- f d) 1)
(EQ (- f d) -1))))))) (EQ 0 0) (EQ 0 0))))))
\end{lstlisting}

Збережемо це у файл, та перевіримо це у Simplify. Отримаємо такий результат:
\begin{lstlisting}[breaklines=true,basicstyle= \ttfamily\footnotesize]
[egor@gauramid Simplify]$ ./Simplify-1.5.4.linux lab2.txt
1: Valid.
\end{lstlisting}

\end{document}
